\documentclass{article}
\usepackage{style}

\title{Экзамен по АиСД 2024/25}
\author{@tiom4eg, $\dots$}
\date{24.03.2025}

\begin{document}

\maketitle

\tableofcontents
\newpage

\section{Асимптотика}

$g(n) \in O(f(n)) \implies \exists C > 0 \forall n : g(n) \leq C \cdot f(n)$

$g(n) \in o(f(n)) \implies \forall C > 0 \exists N : \forall n > N : g(n) < C \cdot f(n)$

$g(n) \in \Omega(f(n)) \implies \exists C > 0 \forall n : g(n) \geq C \cdot f(n)$

$g(n) \in \omega(f(n)) \implies \forall C > 0 \exists N : \forall n > N : g(n) > C \cdot f(n)$

$g(n) \in \Theta(f(n)) \implies g(n) \in \Omega(f(n)) \land g(n) \in O(f(n))$

Амортизированная - достигается в среднем по всем операциям, real-time - гарантированно достигается на каждой операции, ожидаемая - достигается по матожиданию

\section{Теория вероятностей}

Случайные величины $A_1, \dots, A_n$ независимы, если для всех подмножеств $\{A_{i_k}\}$ выполняется $P(\bigcap \limits_{j=1}^{k} A_{i_j}) = \prod \limits_{j=1}^{k} P(A_{i_j})$

Матожидание: $\mathbb{E}[X] = \sum \limits_{\omega} X(\omega) \cdot p(\omega), \ \mathbb{E}[aX + bY] = \sum \limits_{\omega} (a \cdot X(\omega) + b \cdot Y(\omega)) \cdot p(\omega) = \mathbb{E}[aX] + \mathbb{E}[bY] = a \mathbb{E}[X] + b \mathbb{E}[Y], \ \mathbb{E}[XY] = \sum \limits_{\omega} X(\omega) Y(\omega) p_X(\omega) p_Y(\omega)$ (для независимых $X, Y$) $= \sum \limits_{\omega} X(\omega) p_X(\omega) \cdot \sum \limits_{\omega} Y(\omega) p_Y(\omega) = \mathbb{E}[X] \mathbb{E}[Y]$

Дисперсия: $\mathbb{D}[X] = \mathbb{E}[X - \mathbb{E}[X]]^2 = \mathbb{E}[X^2 - 2X\mathbb{E}[X] + \mathbb{E}[X]^2] = \mathbb{E}[X^2] + \mathbb{E}[X]^2 - 2\mathbb{E}[X]\mathbb{E}[X] = \mathbb{E}[X^2] - \mathbb{E}[X]^2$, линейность примерно так же расписывается

\pic{0.3}{static/markov_chebyshev.png}

\section{Quicksort}

Доказательство асимптотики при случайном выборе разделяющей точки (будем считать, что все элементы уникальны): пусть $T_n$ - асимптотика для $n$. $T_n = (n - 1) + \frac{1}{n} \sum \limits_{i=0}^{n-1} T_i + T_{n-1-i} = (n - 1) + \frac{2}{n} \sum \limits_{i=0}^{n-1} T_i \implies nT_n = n(n - 1) + 2 \sum \limits_{i=0}^{n-1} T_i$

$nT_n - (n-1)T_{n-1} = n(n - 1) + 2 \sum \limits_{i=0}^{n-1} T_i - ((n - 1)(n - 2) + 2 \sum \limits_{i=0}^{n-2} T_i) = n(n-1) - (n-1)(n-2) + 2T_{n-1} \implies nT_n = (n+1)T_{n-1} + 2n - 2 \implies \frac{T_n}{n+1} = \frac{T_{n-1}}{n} + \frac{2}{n+1} - \frac{2}{n(n+1)} \leq \frac{T_{n-1}}{n} + \frac{2}{n+1} = \frac{T_{n-2}}{n-1} + \frac{2}{n+1} + \frac{2}{n} - \frac{2}{n(n-1)} \leq \dots \leq \frac{T_1}{2} + \sum \limits_{i=1}^{n+1} \frac{2}{i} = O(n \log n)$

\section{Median of Medians/Quickselect}

По сути, мы хотим находить такую разделяющую точку, что она будет всегда работать достаточно хорошо. Для этого, разобьём все элементы на блоки по 5 элементов, в них отсортируем элементы (по сути, за $O(1)$), далее найдём медиану среди всех медианных элементов в блоках. Заметим, что будет выполнено следующее: в тех блоках, в которых медиана будет меньше медианы медиан, первые три элемента также гарантированно будут меньше, то есть про них мы можем сказать, что они точно окажутся слева от разделяющего элемента. Аналогично, в блоках, в которых медиана больше медианы медиан, последние три элемента гарантированно окажутся справа. Поскольку в обоих случаях блоков будет $\frac{3n}{10}$, получаем, что размер каждой части разбиения будет находиться между $\frac{3n}{10}$ и $\frac{7n}{10}$. Предположим, что каждый раз мы будем попадать в худший из случаев и идти в блок размера $\frac{7n}{10}$. Тогда, $T(n) = T(\frac{n}{5}) + T(\frac{7n}{10}) + O(n) = O(n)$ (по мастер-теореме)

\end{document}
