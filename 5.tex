\section{Кучи}

Инварианты $d$-арной кучи: значение в каждой вершине не больше, чем в любом из её детей, все слои, кроме последнего, заполнены полностью (содержат $k^i$ вершин), в последнем слое заполнен какой-то префикс мест.

Для балансировки кучи будем использовать операции \texttt{sift\_up} и \texttt{sift\_down}. \texttt{sift\_up} будет просто поднимать вершину, пока она больше своего родителя, \texttt{sift\_down} же, наоборот, будет опускать вершину из корня (свапая её с максимальным ребёнком), пока не восстановится инвариант кучи.

Пусть все дети у вершины, которую сейчас просеиваем вниз, образуют в своих поддеревьях корректные $d$-арные кучи. Тогда, либо наша уже корректная (если текущая вершина не больше своих детей), либо мы свапаем её с минимальным ребёнком, тогда инвариант "вершина не больше всех своих детей" сохранится у всех вершин, кроме, возможно, просеиваемой. Примерно так же можно доказать, что \texttt{sift\_up} работает корректно при добавлении нового элемента.

Добавление элемента: добавляем новую вершину на последний слой в первое свободное место, просеиваем её вверх. Удаление минимума - в корень ставим значение какой-либо другой вершины (обычно, последней) и просеиваем вниз. Построение за $O(n)$: закинем все вершины в кучу неважно как, после чего вызовем для всех вершин \texttt{sift\_down} в порядке снизу вверх. Корректность обоснована тем, что при вызове процедуры от некоторой вершины поддеревья её детей уже будут корректными $d$-арными кучами. Поскольку \texttt{sift\_down}, как и \texttt{sift\_up}, работает за высоту кучи, получим сложность $\leq \sum \limits_{i=0}^{\log_{d} n} (i + 1)\frac{n}{d^i} = O(n)$.
